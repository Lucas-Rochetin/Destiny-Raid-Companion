\chapter{Méthodologie et organisation}

\section{Gestion de projet avec GitHub}

\textbf{Votre approche GitHub :} \textit{Le projet Destiny 2 Raid Companion est géré entièrement sur GitHub avec une approche Agile adaptée au développement en solo. L'organisation repose sur GitHub Projects pour le suivi des tâches, les Milestones pour la planification temporelle, et un workflow Git Flow modifié pour assurer la qualité du code.}

\subsection{Adaptation de la méthode Agile au contexte}

\textbf{Pourquoi l'Agile est adapté à ce projet :}
\begin{itemize}
    \item \textbf{Projet innovant :} Besoin de s'adapter aux retours utilisateurs rapidement
    \item \textbf{API tierce complexe :} Nécessité d'itérer sur l'intégration Bungie API
    \item \textbf{Développement solo :} Flexibilité pour ajuster les priorités selon les blocages
    \item \textbf{Validation continue :} MVP à tester rapidement avec la communauté Destiny 2
\end{itemize}

\textbf{Justification de GitFlow pour un projet solo :} Le modèle GitFlow est adapté même en solo car il permet d'isoler les fonctionnalités (feature branches), de préparer les releases (release branches), et de gérer les correctifs urgents (hotfix branches) sans polluer la branche principale. Cette discipline prépare également l'arrivée potentielle de contributeurs supplémentaires et facilite le rollback en cas de problème avec l'API Bungie.

\subsection{Rituels Agile et leur mise en œuvre avec objectifs métier}

\textbf{Objectif métier de chaque rituel :}
\begin{itemize}
    \item \textbf{Daily Standup :} Identifier rapidement les blocages techniques (ex: rate limiting API Bungie) et ajuster les priorités du jour
    \item \textbf{Sprint Planning :} Aligner les développements avec la roadmap MVP et anticiper les dépendances API externes
    \item \textbf{Sprint Review :} Valider la valeur métier produite avec des tests utilisateurs (ex: guides validés par des joueurs expérimentés)
    \item \textbf{Sprint Retrospective :} Améliorer le processus pour réduire le temps de cycle et augmenter la qualité
\end{itemize}

\begin{center}
\begin{tabular}{|p{3cm}|p{3.5cm}|p{4cm}|}
\hline
\textbf{Rituel} & \textbf{Fréquence} & \textbf{Objectif métier et mise en œuvre} \\
\hline
\textbf{Daily Standup} & Quotidien (10 min) & Identifier rapidement les blocages (rate limiting API Bungie) et ajuster les priorités. Mise à jour GitHub Projects avec statut réel \\
\hline
\textbf{Sprint Planning} & Tous les 15 jours & Aligner les développements avec la roadmap MVP et anticiper les dépendances API. Sélection issues, estimation, définition objectifs sprint \\
\hline
\textbf{Sprint Review} & Fin de sprint & Valider la valeur métier avec tests utilisateurs (guides validés par Sarah, leader expérimentée). Démonstration fonctionnalités, recueil retours \\
\hline
\textbf{Sprint Retrospective} & Fin de sprint & Améliorer le processus pour réduire le temps de cycle. Identification points d'optimisation, actions correctives \\
\hline
\end{tabular}
\end{center}
\vspace{0.5cm}
\textbf{Analyse des métriques Agile et interprétation :}
\begin{itemize}
    \item \textbf{Force :} Vélocité stable à 8-12 points/sprint montre une bonne régularité de développement malgré la complexité de l'API Bungie
    \item \textbf{Risque :} Cycle time de 4.2 jours révèle que certaines issues sont trop larges (nécessite meilleur découpage)
    \item \textbf{Amélioration :} Les 3 bugs majeurs résolus en 2.5 jours montrent une bonne réactivité, mais le taux de bugs (15\%) indique besoin de plus de tests automatisés
    \item \textbf{Tendance positive :} Lead time réduit de 7 à 5 jours montre une amélioration de l'efficacité globale
\end{itemize}
\vspace{0.5cm}
\textbf{Métriques de suivi avec interprétation :}
\begin{itemize}
    \item \textbf{Vélocité :} 8-12 story points par sprint (stable, bonne prévisibilité)
    \item \textbf{Taux de complétion :} > 85\% des tâches par sprint (excellente fiabilité)
    \item \textbf{Bugs ouverts/fermés :} Ratio < 0.5 (2 bugs fermés pour 1 ouvert - bonne réactivité)
    \item \textbf{Lead time :} < 5 jours pour les issues critiques (efficace pour les blocages)
    \item \textbf{Cycle time :} 4.2 jours en moyenne (besoin d'amélioration du découpage)
\end{itemize}

\textbf{Colonnes du tableau Kanban :}
\begin{itemize}
    \item \textbf{Backlog :} Fonctionnalités à développer (triées par priorité métier)
    \item \textbf{Sprint Backlog :} Tâches sélectionnées pour le sprint courant (alignées roadmap)
    \item \textbf{To Do :} Tâches prêtes pour le développement (DoD vérifié)
    \item \textbf{In Progress :} Tâches en cours (WIP limit: 2 - focus qualité)
    \item \textbf{Review :} Code en attente de validation (tests, revue de code)
    \item \textbf{Done :} Fonctionnalités livrées et validées (critères d'acceptation remplis)
\end{itemize}

\subsection{User Stories et estimation de temps}

\textbf{Système d'estimation justifié :} Les story points sont estimés selon trois critères : complexité technique (intégration API Bungie), incertitude (données non documentées), et effort de test. Par exemple, une fonctionnalité touchant l'API Bungie reçoit automatiquement +2 points pour l'incertitude. La complexité est évaluée sur une échelle de 1 à 8, avec des points bonus pour les dépendances externes.

\begin{itemize}
    \item \textbf{1 point :} Tâche simple (< 1 jour) - Configuration, corrections mineures
    \item \textbf{3 points :} Tâche moyenne (1-2 jours) - Composants frontend simples
    \item \textbf{5 points :} Tâche complexe (3-4 jours) - Intégration API avec gestion d'erreurs
    \item \textbf{8 points :} Tâche très complexe (> 5 jours) - Mécaniques de matching algorithmique
\end{itemize}

\textbf{Reliement User Stories → Sprints → Roadmap :} Les US \#101 (Authentification Bungie) et \#102 (Guides interactifs) représentent 70\% du périmètre MVP v1.0 prévu pour le 15 mars 2026. Ces US sont décomposées en 15 sous-tâches réparties sur 3 sprints, avec des dépendances claires (backend → frontend → tests).

\textbf{User stories avec estimations et alignement roadmap :}
\begin{figure}[h]
    \centering
    \includegraphics[width=0.8\textwidth]{assets/capture_US.png}
    \caption{User stories}
    \label{fig:contexte}
\end{figure}

\section{Versioning GitHub et conventions}

Le versioning GitHub suit le modèle Git Flow avec des branches spécialisées pour chaque type de développement. Cette approche est particulièrement adaptée au développement solo car elle permet d'isoler les fonctionnalités, de tester indépendamment les intégrations API Bungie, et de préparer les releases sans interrompre le développement principal. Elle facilite également le rollback en cas de problème avec l'API externe.

\subsection{CONTRIBUTING.md et normalisation}

\textbf{Contenu du CONTRIBUTING.md :}
\begin{itemize}
    \item \textbf{Environnement :} Setup du projet, pré-requis, installation avec variables API Bungie
    \item \textbf{Conventions de code :} ESLint, Prettier, standards React/Node.js avec règles spécifiques API
    \item \textbf{Workflow Git :} Processus de création de branches, commits, PR avec validation
    \item \textbf{Testing :} Comment exécuter les tests, couverture attendue, mocks API Bungie
    \item \textbf{Code Review :} Checklist pour la revue de code avec focus sécurité OAuth
\end{itemize}

\subsection{Conventions de branches}

\begin{center}
\begin{tabular}{|p{3cm}|p{8cm}|}
\hline
\textbf{Type de branche} & \textbf{Convention de nommage et justification} \\
\hline
Feature & \texttt{feature/nom-fonctionnalite} ou \texttt{feature/issue-\#123} - Isolation pour développement et test \\
\hline
Bugfix & \texttt{fix/description-bug} ou \texttt{fix/issue-\#456} - Correction ciblée sans affecter autres features \\
\hline
Hotfix & \texttt{hotfix/description-urgente} - Pour correctifs critiques (ex: API Bungie cassée) \\
\hline
Release & \texttt{release/v1.0.0} - Préparation release avec tests intensifs \\
\hline
Documentation & \texttt{docs/sujet-documentation} - Mise à jour documentation sans risque code \\
\hline
\end{tabular}
\end{center}

\subsection{Conventions de commits}

\textbf{Schéma Git Flow adapté au projet solo :}
\begin{verbatim}
main (protected) ---------
  |                      |          
  |                      |          
  |                      |
  +-- develop (protected) ---
       |                    |          
       |                    |          
       |                    |
       +-- feature/bungie-oauth
       +-- feature/raid-guides
       +-- fix/login-validation
\end{verbatim}
\vspace{0.5cm}
\textbf{Exemple réel de PR respectant le DoD (PR \#45) :}
\begin{itemize}
    \item \textbf{Fonctionnalité :} Ajout guide interactif "Vault of Glass"
    \item \textbf{Tests :} 12 tests unitaires passants (guides.spec.js), coverage 92\%
    \item \textbf{Intégration :} Tests d'intégration API Bungie validés
    \item \textbf{Review :} Auto-review avec checklist complétée
    \item \textbf{Déploiement :} Build CI/CD réussi, déployé sur staging
    \item \textbf{Documentation :} Guide utilisateur mis à jour, commentaires code ajoutés
    \item \textbf{Sécurité :} Validation OAuth et sanitization des données
\end{itemize}
\vspace{0.5cm}
\textbf{Conventions de commit (Conventional Commits) :}
\begin{lstlisting}
feat: add OAuth Bungie authentication system
fix: resolve login token expiration issue
docs: update API integration guide
test: add unit tests for user service
refactor: improve raid guide component structure
style: format code with prettier
chore: update dependencies to latest versions
\end{lstlisting}
\vspace{0.5cm}
\textbf{Definition of Done (DoD) appliqué concrètement :}
\begin{itemize}
    \item \textbf{Tests :} Couverture > 80\%, tests unitaires et d'intégration passants
    \item \textbf{Code Review :} Au moins une review effectuée (auto-review en solo)
    \item \textbf{CI/CD :} Pipeline GitHub Actions réussie (build, test, scan)
    \item \textbf{Déploiement :} Déployé sur environnement de test et validé
    \item \textbf{Documentation :} Code documenté, changelog mis à jour
    \item \textbf{Sécurité :} Scan de vulnérabilités passé, secrets protégés
    \item \textbf{Performance :} Tests de performance validés pour l'API Bungie
\end{itemize}

\section{Planification et outils de suivi}

La planification combine une roadmap GitHub pour la vision macro et GitHub Projects pour le suivi opérationnel. Cette approche duale optimise la coordination entre la planification stratégique et l'exécution tactique, particulièrement importante avec les délais imprévisibles de l'API Bungie.

\subsection{GitHub Project et Roadmap}

\textbf{Structure du GitHub Project :}
\begin{itemize}
    \item \textbf{Vue Kanban :} Suivi visuel de l'état des tâches avec WIP limits
    \item \textbf{Filtres :} Par label, milestone, assigné, statut, priorité API
    \item \textbf{Automatisations :} Changement de statut basé sur les PR/issues (ex: auto-move to Review)
    \item \textbf{Vues personnalisées :} Tableau de bord pour daily standup avec métriques clés
\end{itemize}
\vspace{0.5cm}
\textbf{Roadmap GitHub avec dépendances API :}
\begin{itemize}
    \item \textbf{Visibilité :} Roadmap publique pour transparence avec communauté
    \item \textbf{Milestones :} Dates cibles ajustables selon disponibilité API Bungie
    \item \textbf{Dépendances :} Liens explicites entre fonctionnalités et endpoints API
    \item \textbf{Suivi progression :} Avancement visuel avec indicateurs de risque API
\end{itemize}
\vspace{0.5cm}
\textbf{Extrait de roadmap GitHub aligné avec User Stories :}
\begin{verbatim}
Phase 1: MVP (Oct 2025 - Mars 2026) [US #101-#104]
+-- Sprint 1: Setup & Auth (4 semaines) [US #101]
    +-- Environnement dev (1 semaine) [Issue #1]
    +-- Authentification Bungie (2 semaines) [Issue #2, dépendance API]
    +-- Base de données (1 semaine) [Issue #3]
    
+-- Sprint 2: Core Features (6 semaines) [US #102-#103]
    +-- Guides interactifs (3 semaines) [Issue #4, dépendance API]
    +-- Gestion escouades (2 semaines) [Issue #5]
    +-- Calendrier raids (1 semaine) [Issue #6]

Phase 2: Version 1.0 (Avril - Mai 2026) [US #105-#107]
+-- Sprint 3: Gamification & Analytics [Issue #7]
+-- Sprint 4: Finalisation & Déploiement [Issue #8]
\end{verbatim}
\vspace{0.5cm}
\textbf{Configuration GitHub Projects :}
\begin{itemize}
    \item \textbf{Colonnes :} Backlog, Sprint Planning, In Progress, Review, Done
    \item \textbf{WIP Limits :} 2 tâches max en cours par développeur (focus qualité)
    \item \textbf{Policies :} PR obligatoire pour merge en develop avec DoD vérifié
    \item \textbf{Automation :} Mise à jour automatique des statuts via GitHub Actions
\end{itemize}

\subsection{Liaison User Stories - Tests - Milestones}

\textbf{Intégration complète dans GitHub :}
\begin{itemize}
    \item \textbf{User Stories :} Créées comme issues avec template dédié et critères d'acceptation
    \item \textbf{Tests :} Issues liées pour les scénarios de test avec données API Bungie
    \item \textbf{Milestones :} Regroupement logique par version avec dépendances explicites
    \item \textbf{Labels :} Complexité (S, M, L, XL), type (bug, feature, docs), priorité API
\end{itemize}
\vspace{0.5cm}
\textbf{Workflow de validation avec exemple concret :}
\begin{enumerate}
    \item Issue \#101 créée avec critères d'acceptation spécifiques à l'API Bungie
    \item Branche \texttt{feature/bungie-oauth} développée avec tests associés
    \item Pull Request \#45 avec validation des tests automatisés (12 tests passants)
    \item Revue de code auto-effectuée avec checklist DoD
    \item Merge et déploiement automatique en environnement de test
    \item Validation manuelle avec compte Bungie de test
\end{enumerate}

\section{Estimation de temps et planification}

\textbf{Estimation globale :} \textit{Le projet est estimé à 65 jours de travail effectif répartis sur 8 mois (octobre 2025 à mai 2026), incluant 20\% de marge pour les imprévus liés à l'API Bungie. Cette estimation couvre le développement, les tests d'intégration API, et le déploiement.}
\vspace{0.5cm}
\textbf{Estimation détaillée avec justification technique :}
\begin{longtable}{|p{3cm}|p{2.5cm}|p{1.5cm}|p{1.5cm}|p{3cm}|}
\toprule
\textbf{Fonctionnalité} & \textbf{Phase} & \textbf{SP} & \textbf{Jours} & \textbf{Justification estimation} \\
\midrule
Environnement de développement & Setup & 3 & 3 & Standard, pas de risque \\
\hline
Authentification Bungie OAuth & Backend & 5 & 5 & Complexité API OAuth + gestion tokens \\
\hline
Base de données PostgreSQL & Backend & 3 & 3 & Standard, schéma validé \\
\hline
API Gestion escouades & Backend & 5 & 5 & Logique métier complexe, tests \\
\hline
Guides interactifs raids & Frontend & 8 & 8 & UI complexe, intégration données API \\
\hline
Calendrier raids & Frontend & 5 & 5 & Composants React avancés \\
\hline
Profils joueurs & Frontend & 3 & 3 & Simple affichage données \\
\hline
Intégration API Destiny 2 & Intégration & 8 & 8 & Risque élevé, documentation API limitée \\
\hline
Système de badges & Fonctionnalité & 5 & 5 & Logique métier, tests gamification \\
\hline
Tests unitaires et intégration & Qualité & 8 & 8 & Couverture élevée requise pour API \\
\hline
Tests E2E & Qualité & 5 & 5 & Scénarios utilisateurs complexes \\
\hline
Dockerisation & Déploiement & 3 & 3 & Standard, configuration API \\
\hline
CI/CD & Déploiement & 5 & 5 & Pipeline complexe avec tests API \\
\hline
Documentation technique & Livraison & 3 & 3 & Documentation API spécifique \\
\midrule
\textbf{Total} & & \textbf{70} & \textbf{70 jours} & \\
\textbf{Avec marge 20\% (risques API)} & & \textbf{84} & \textbf{84 jours} & \\
\bottomrule
\end{longtable}

\subsection{Métriques de suivi et amélioration continue}

\textbf{Métriques collectées avec objectifs d'amélioration :}
\begin{itemize}
    \item \textbf{Vélocité :} Objectif: stabiliser à 10-12 points/sprint (actuel: 8-12)
    \item \textbf{Burndown chart :} Détection précoce des retards liés à l'API Bungie
    \item \textbf{Lead time :} Objectif: réduire de 5 à 4 jours via meilleur découpage
    \item \textbf{Cycle time :} Objectif: réduire de 4.2 à 3.5 jours via automation tests
    \item \textbf{Taux de bugs :} Objectif: réduire de 15\% à 10\% via plus de tests unitaires
\end{itemize}
\vspace{0.5cm}
\textbf{Amélioration continue basée sur données :}
\begin{itemize}
    \item \textbf{Rétrospectives :} Actions concrètes comme "ajouter mocks API pour tests"
    \item \textbf{Ajustements :} Réduction taille des User Stories basée sur cycle time
    \item \textbf{Qualité code :} Augmentation couverture tests de 80\% à 85\% ciblée
    \item \textbf{Satisfaction :} Retours utilisateurs directs intégrés dans backlog
\end{itemize}

\section{Liens utiles}

\begin{itemize}
    \item GitHub Project: \url{https://github.com/xxx/projects/1}
    \item CONTRIBUTING.md: \url{https://github.com/xxx/CONTRIBUTING.md}
    \item GitHub Flow/PRs: \url{https://docs.github.com/pull-requests}
    \item Git Flow: \url{https://bit.ly/gitflow-atlassian}
    \item GitHub Projects: \url{https://bit.ly/github-projects}
    \item GitHub Roadmap: \url{https://bit.ly/github-roadmap}
    \item GitHub Milestones: \url{https://bit.ly/github-milestones}
    \item User Stories: \url{https://www.mountaingoatsoftware.com/agile/user-stories}
    \item Estimation de temps: \url{https://bit.ly/time-estimation}
    \item Conventional Commits: \url{https://www.conventionalcommits.org}
\end{itemize}