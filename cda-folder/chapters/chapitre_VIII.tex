\chapter{Déploiement et CI/CD (suite)}

\subsection{Script de Déploiement Blue-Green}

\begin{lstlisting}[language=bash]
#!/bin/bash
# scripts/deploy-blue-green.sh
set -e
ENVIRONMENT=$1
IMAGE_TAG=$2
TIMESTAMP=$(date +%Y%m%d-%H%M%S)

echo "Starting blue-green deployment to $ENVIRONMENT with image tag $IMAGE_TAG"

# Déterminer les services actuels et nouveaux
CURRENT_COLOR=$(docker service ls --filter name=destiny-app --format "{{.Name}}" | grep -o "blue\|green" || echo "blue")
if [ "$CURRENT_COLOR" = "blue" ]; then
    ACTIVE_SERVICE="destiny-app-blue"
    NEW_SERVICE="destiny-app-green"
    NEW_COLOR="green"
else
    ACTIVE_SERVICE="destiny-app-green"
    NEW_SERVICE="destiny-app-blue"
    NEW_COLOR="blue"
fi

echo "Current active service: $ACTIVE_SERVICE"
echo "Deploying new service: $NEW_SERVICE ($NEW_COLOR)"

# Créer le nouveau service
docker service create \
    --name $NEW_SERVICE \
    --with-registry-auth \
    --replicas 1 \
    --update-delay 10s \
    --update-parallelism 1 \
    --update-order start-first \
    --restart-condition on-failure \
    --restart-delay 5s \
    --restart-max-attempts 3 \
    --limit-memory 512M \
    --reserve-memory 256M \
    --env NODE_ENV=production \
    --env BUNGIE_API_KEY=$BUNGIE_API_KEY \
    --env DATABASE_URL=$DATABASE_URL \
    --env REDIS_URL=$REDIS_URL \
    --secret bungie_api_key \
    --secret database_url \
    --network destiny-network \
    ghcr.io/username/destiny-raid-companion:$IMAGE_TAG

# Attendre que le nouveau service soit healthy
echo "Waiting for new service to be healthy..."
for i in {1..30}; do
    HEALTH=$(docker service inspect $NEW_SERVICE --format "{{.UpdateStatus.State}}")
    if [ "$HEALTH" = "completed" ]; then
        echo "New service is healthy!"
        break
    fi
    echo "Waiting... ($i/30)"
    sleep 5
done

if [ "$HEALTH" != "completed" ]; then
    echo "ERROR: New service failed to become healthy"
    docker service rm $NEW_SERVICE
    exit 1
fi

# Mettre à jour le load balancer
sed -i "s/destiny-app-$CURRENT_COLOR/destiny-app-$NEW_COLOR/g" /etc/nginx/upstreams.conf
nginx -s reload

# Scale up le nouveau service, scale down l'ancien
docker service scale $NEW_SERVICE=2
docker service scale $ACTIVE_SERVICE=0
sleep 30  # Attendre le drainage des connexions

# Supprimer l'ancien service
docker service rm $ACTIVE_SERVICE

# Mettre à jour les labels
docker service update $NEW_SERVICE --label-add deploy.color=$NEW_COLOR --label-add deploy.timestamp=$TIMESTAMP

echo "Blue-green deployment completed successfully!"
echo "Active service: $NEW_SERVICE"
echo "Previous service: $ACTIVE_SERVICE (removed)"
\end{lstlisting}

\section{Infrastructure et Monitoring}

L'infrastructure de Destiny Raid Companion est déployée sur une plateforme cloud avec auto-scaling basé sur la charge. Le monitoring complet inclut des métriques applicatives (performance des guides, succès des appels API Bungie), des métriques business (nombre d'escouades créées, guides consultés), et des alertes automatiques.

\subsection{Configuration Prometheus pour Métriques Personnalisées}

\begin{lstlisting}[language=yaml]
# monitoring/prometheus.yml
global:
  scrape_interval: 15s
  evaluation_interval: 15s

rule_files: ["alerts.yml"]

scrape_configs:
  - job_name: 'destiny-app'
    static_configs: [{ targets: ['destiny-app:3000'] }]
    metrics_path: '/api/metrics'
    params: { format: ['prometheus'] }
  
  - job_name: 'bungie-api'
    static_configs: [{ targets: ['destiny-app:3000'] }]
    metrics_path: '/api/bungie/metrics'
  
  - job_name: 'postgres'
    static_configs: [{ targets: ['postgres-exporter:9187'] }]
  
  - job_name: 'redis'
    static_configs: [{ targets: ['redis-exporter:9121'] }]
  
  - job_name: 'node'
    static_configs: [{ targets: ['node-exporter:9100'] }]

alerting:
  alertmanagers: [{ static_configs: [{ targets: ['alertmanager:9093'] }] }]
\end{lstlisting}

\subsection{Métriques Personnalisées Destiny}

\begin{lstlisting}[language=JavaScript]
// src/metrics/destinyMetrics.js
const client = require('prom-client');
const register = new client.Registry();

// Métriques API Bungie
const bungieApiCalls = new client.Counter({
  name: 'destiny_bungie_api_calls_total',
  help: 'Total number of Bungie API calls',
  labelNames: ['endpoint', 'method', 'status']
});

const bungieApiLatency = new client.Histogram({
  name: 'destiny_bungie_api_latency_seconds',
  help: 'Bungie API response latency',
  labelNames: ['endpoint'],
  buckets: [0.1, 0.5, 1, 2, 5]
});

// Métriques guides
const guideViews = new client.Counter({
  name: 'destiny_guide_views_total',
  help: 'Total number of guide views',
  labelNames: ['raid', 'difficulty']
});

// Métriques escouades
const squadsCreated = new client.Counter({
  name: 'destiny_squads_created_total',
  help: 'Total number of squads created',
  labelNames: ['raid']
});

// Métriques utilisateurs
const activeUsers = new client.Gauge({
  name: 'destiny_active_users',
  help: 'Number of currently active users'
});

// Enregistrement des métriques
[ bungieApiCalls, bungieApiLatency, guideViews, squadsCreated, activeUsers ]
  .forEach(metric => register.registerMetric(metric));

// Middleware de collecte
const metricsMiddleware = (req, res, next) => {
  const start = Date.now();
  res.on('finish', () => {
    const duration = (Date.now() - start) / 1000;
    
    if (req.path.includes('/api/bungie/')) {
      const endpoint = req.path.split('/').pop();
      bungieApiCalls.inc({ endpoint, method: req.method, status: res.statusCode });
      bungieApiLatency.observe({ endpoint }, duration);
    }
    
    if (req.path.includes('/guides/') && req.method === 'GET') {
      const raid = req.params.raidId;
      guideViews.inc({ raid });
    }
  });
  next();
};

module.exports = { metricsMiddleware, register };
\end{lstlisting}

\subsection{Alertes Personnalisées pour Destiny Companion}

\begin{lstlisting}[language=yaml]
# monitoring/alerts.yml
groups:
  - name: destiny-api-alerts
    rules:
      - alert: HighBungieAPILatency
        expr: histogram_quantile(0.95, rate(destiny_bungie_api_latency_seconds_bucket[5m])) > 2
        for: 5m
        labels: { severity: warning, service: bungie-api }
        annotations:
          summary: "High latency on Bungie API calls"
          description: "Bungie API latency P95 is {{ $value }}s (threshold: 2s)"
      
      - alert: BungieAPIFailureRate
        expr: rate(destiny_bungie_api_calls_total{status=~"5.."}[5m]) / rate(destiny_bungie_api_calls_total[5m]) > 0.05
        for: 2m
        labels: { severity: critical, service: bungie-api }
        annotations:
          summary: "High Bungie API failure rate"
          description: "Bungie API failure rate is {{ $value | humanizePercentage }} (threshold: 5%)"
      
      - alert: GuidePopularityDrop
        expr: rate(destiny_guide_views_total[1h]) / rate(destiny_guide_views_total[1h] offset 1d) < 0.5
        for: 1h
        labels: { severity: warning, service: guides }
        annotations:
          summary: "Significant drop in guide views"
          description: "Guide views dropped by {{ $value | humanizePercentage }} compared to yesterday"
      
      - alert: SquadCreationFailure
        expr: rate(destiny_squads_created_total[10m]) == 0
        for: 30m
        labels: { severity: critical, service: squads }
        annotations:
          summary: "No squads created in 30 minutes"
          description: "Squad creation service may be down"
\end{lstlisting}

\subsection{Tableau de Bord Grafana - Métriques Clés}

\begin{center}
\begin{tabular}{|p{3.5cm}|p{3cm}|p{3cm}|}
\hline
\textbf{Dashboard} & \textbf{Métriques Principales} & \textbf{Seuils d'Alerte} \\
\hline
Performance API Bungie & Latence P95, taux succès, quota usage & Latence > 2s, succès < 95\%, quota > 80\% \\
\hline
Utilisation Guides & Vues par raid, temps moyen, popularité & Abandon > 50\%, temps < 30s \\
\hline
Activité Escouades & Créations, sessions réussies, taille moyenne & Sessions échouées > 20\% \\
\hline
Performance Applicative & CPU, mémoire, débit, erreurs & CPU > 80\%, mémoire > 85\%, erreurs > 5\% \\
\hline
Croissance Utilisateurs & Nouveaux inscrits, actifs, rétention & Churn > 10\%, croissance < 5\% \\
\hline
\end{tabular}
\end{center}

\section{Documentation Opérationnelle}

La documentation opérationnelle pour Destiny Raid Companion inclut des runbooks détaillés pour les scénarios courants, des procédures d'urgence, et des guides de dépannage.
